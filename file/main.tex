\documentclass[a4paper,ngerman, 11pt]{report}
\usepackage{geometry}
\usepackage{setspace}
\usepackage{blindtext}
\usepackage{picture}
\usepackage{graphicx}
\usepackage[ngerman]{babel}
\usepackage[autostyle=true]{csquotes}
\usepackage[backend=biber]{biblatex}

\addbibresource{referenzen.bib}
\renewcommand{\familydefault}{\sfdefault}


\begin{document}
    \begin{titlepage}  
        \begin{center}
            \LARGE Geschwister-Scholl-Gymnasium Velbert\par
            \vspace{1cm}
            \large Facharbeit Informatik \par
            \vspace{1.5cm}
            {\huge\bfseries Die Enigma Verschlüsselung\par}
            \large Mattis Jung, Q1 25/26\par
        \end{center}
        \vfill

        \large 09.01.2026 - 23.02.2026\par
        6 Wochen
    \end{titlepage}

\newpage
\tableofcontents 
\newpage

\begin{onehalfspace}
\newgeometry{
    left=4cm,
    right=2cm,
    top=2.5cm,
    bottom=4cm,
    footskip=2cm,
    includefoot
}
\pagestyle{plain}

\newpage
\section{Einleitung}

\newpage
\section{Die Enigma-Maschiene}

\subsection{Was die Enigma-Maschiene ist}

    Die Enigma-Maschiene ist eine Chiffriermaschiene, die 
    durch die Nutzung von einem Streckerbrett und mehreren 
    Walzen eine Nachricht Buchstabe für Buchstabe verschlüsseln.
    


    \subsection{Wie die Enigma-Maschiene entstand}

    \subsection{Wie die Enigma-Maschiene funktioniert}
    Dabei sind 20 der 26 Buchstaben des deutschen Alphabets 
    \emph{gesteckert} und die restlichen 6 nicht.

    \subsection{Was ist das Problem der Enigma-Maschiene}

\newpage
\section{Meine Enigma Implementation}

\subsection{Wie funktioniert diese Implementation?}

\newpage
\section{Die perfekte Enigma-Maschiene}

\subsection{Was ist anders an dieser Enigma-Maschiene?}


\end{onehalfspace}
\newpage
\printbibliography
\end{document}