\documentclass[german,12pt,a4paper]{scrartcl}
\usepackage[left=35mm,right=25mm, nohead]{geometry} 	%-- Seitenränder
\usepackage[font=small, format=hang]{caption}		%-- z.B. Bildunterschriften 
\setlength{\captionmargin}{30pt}						%-- 30pt links und rechts des Bildes
\usepackage[utf8]{inputenx}							%-- UTF8-Zeichensatz
\usepackage[german]{babel}									%-- Deutsche Satz-/Sonstwas-zeichen ...
\usepackage[T1]{fontenc}								%-- "Echte Umlaute"... Silbentrennung.
\usepackage{amsmath}								%-- Textsatz für Mathematik
\usepackage{amsfonts}								%-- american mathematical society
\usepackage{amssymb}								%-- ...
\usepackage{amsthm}									%-- Definition, Satz, Beweis, etc...
\usepackage{graphicx}								%-- Xtended graphic
\usepackage{subfigure}
\usepackage{color}									%-- Farbe
\usepackage{framed}									%-- Für Frames, Schatten, etc.
\usepackage{isotope}								%-- Darstellung Kernphysik
\usepackage{epigraph}								%-- Sprichwortumgebung
\usepackage{polynom}
\usepackage{picture}
\usepackage{icomma}
\usepackage{xcolor}
\usepackage{eurosym}
\usepackage{tcolorbox}
\usepackage{url}
\usepackage{lipsum} % Für Dummytext

% ----------- Java und andere Sachen ----------
\usepackage{listings}
\definecolor{dkgreen}{rgb}{0,0.6,0}
\definecolor{gray}{rgb}{0.5,0.5,0.5}
\definecolor{mauve}{rgb}{0.58,0,0.82}

\lstset{frame=tb,
    language=Java,
    aboveskip=3mm,
    belowskip=3mm,
    showstringspaces=false,
    columns=flexible,
    basicstyle={\small\ttfamily},
    numbers=none,
    numberstyle=\tiny\color{gray},
    keywordstyle=\color{blue},
    commentstyle=\color{dkgreen},
    stringstyle=\color{mauve},
    breaklines=true,
    breakatwhitespace=true,
    tabsize=3
}

\usepackage{biblatex}
\addbibresource{literatur.bib}

% --------- Darstellung, Stil, etc. des Literaturverzeichnisses ----------
%\usepackage[sectionbib,square]{natbib}
%\usepackage[square,sort,comma,numbers,super]{natbib}
%\bibliographystyle{plainnat} % abbrvnat, plainnat, unsrtnat
%\usepackage[superscript]{cite}


% --------- Schulfarben festlegen --------------
\definecolor{titlepagecolor}{rgb}{0.7,0.8,.8}
\definecolor{namecolor}{rgb}{.24,.32,.56} 

% ---------------------------------------------------------------------------------
% -																				  -
% -								Hier fängt das Dokument an ...					  -
% -																				  -
% ---------------------------------------------------------------------------------
\begin{document}

% Erstellen der Titelseite
\begin{titlepage}
\newgeometry{left=8.5cm} 
\pagecolor{titlepagecolor}
\noindent
\includegraphics[width=11cm]{bilder/header.png}\\[-1em]
\color{white}
\par
\noindent
\textbf{\textsf{Geschwister-Scholl-Gymnasium}} \textcolor{namecolor}{\textsf{Velbert}}
\vfill
\noindent
{\huge \textsf{Facharbeiten schreiben mit \LaTeX{} }}
\vskip\baselineskip
\noindent
{\Large \textsf{Mattis Jung}}
\vskip\baselineskip
\noindent
\textsf{\today}
\end{titlepage}

\restoregeometry % restores the geometry
\pagecolor{white}
\nopagecolor% Use this to restore the color pages to white

% -----------------------------------------------------------------------------------
% Stil, in dem die Seiten gezählt werden 
\pagenumbering{arabic}


\begin{abstract}
Dies ist ein kleiner Wegweiser für das Schreiben einer Facharbeit. Der 
Schwerpunkt liegt hier auf der Verwendung von \LaTeX{}.

Am einfachsten ist es vermutlich, dieses Dokument als PDF anzuschauen 
und nebenbei im Quelltext nachzuschauen, wie dies und das umgesetzt wurde. 
Parallel dazu füllt man sein eigenes Dokument (\emph{Facharbeit.tex}).
\end{abstract}


% Hier wird automatisch das Inhaltsverzeichnis generiert
\newpage
\tableofcontents 
\newpage


% -------- Hier beginnt der eigentliche Inhalt ------------
\section{Einleitung}
\epigraph{Das Lernen macht stets dann Verdruß’, wenn man’s nicht will, 
es aber muss.}{Heinz Erhardt (1909 - 1979)}

\section{Ein bißchen zu \LaTeX\ und zur Typographie}
\subsection{Typographische Feinheiten, die viele nicht kennen...}
Hier mal der Unterscheid zwischen einem ganzen und einem halben\cite{Patentschrift}
(richtige Version) Leerzeichen vor Einheiten:\\
100 m oder auch 100m sehen weniger schön aus, als 100\,m. 
Ein Eurosymbol erzeugt man durch einen Befehl (siehe Quelltext): 
\EUR{22}... 
Gleiches gilt für ein Prozentzeichen \%, was ja eigentlich den Text 
im Quelltext auskommentieren würde... 
Bei Abkürzungen kommt ebenfalls das halbe Leerzeichen zur Anwendung.
Man schreibt z.\,B. (nicht z.B. oder z. B.) 70\,\% statt 70\% oder 
gar 70 \% (halbes Leerzeichen). Siehe Quelltext.


Um etwas wichtiges hervorzuheben, verwendet man übrigens nicht den 
aufdringlichen (wenn auch leider von vielen verwendeten)
\textbf{Fettdruck} sondern die dezente \emph{italienische} Variante! 
Benötigt man mehr, gibt es auch noch: \textsf{etwas ohne Serifen} 
(gesehen?) oder \texttt{Schreibmaschinenschrift}. Natürlich geht noch 
viel mehr allerdings sollte man es vermeiden Schriftarten zu mischen, 
wenn man nicht ganz genau weiß was man tut. Die italienische Variante 
reicht in der Regel.

Absätze sind am Anfang immer etwas eingerückt und werden im Quelltext 
durch eine Leerzeile erzeugt. 

Sehr empfehlenswert ist auch die kleine PDF „typokurz\cite{typo}“.

\subsection{Ein paar Beispiele für den Einsatz von \LaTeX}
\LaTeX, ausgesprochen „Latech“ funktioniert nach dem Prinzip des 
Textsatzes, Word und andere nach dem Wortsatzverfahren.

Word funktioniert nach dem Prinzip: „What You See Is What You Get“ 
(WYSIWIG) - das fertige Ergebnis ist von Anfang an auf dem 
PC-Bildschirm zu erkennen. Die \LaTeX-Community setzt diesem Slogan 
ihr eigenes Motto entgegen: WYGIWYM steht für „What You Get Is What 
You Mean“ - Man bekommt das, was man auch wirklich beabsichtigt hat.

\subsubsection{Darstellung Chemie}

$\isotope{Po} --- \isotope[56]{Fe} --- \isotope[13][6]{C}$

$\isotope{n} + \isotope{H} \to \isotope{D} +
\gamma(2.2\,\mathrm{MeV})$

$\isotope[13][6]{C} + \alpha(5.314\,\mathrm{MeV}) \to
\isotope[16]{O}^{**} + n$

\subsubsection{Darstellung Mathematik}
\begin{align}
\int_0^6 x^2 dx = 72
\end{align}
Umgebungen, wie hier die \emph{align}-Umgebung können zusätzlich mit 
einem Sternchen versehen werden, wie beispielsweise im nächsten Fall, 
wodurch die Formelnummerierung (auf die übrigens wieder referenziert 
werden können) ausgeschaltet werden kann. 

Hier werden mehrere Formeln untereinander am Gleichheitszeichen 
ausgerichtet dargestellt.
\begin{align*}
\int_0^6 x^2 dx &= 72 \\ 		% Der doppelte Backslash ist für den Zeilenumbruch zuständig, 
								% das &-Zeichen, gibt an, woran die Formeln
f(x)	&= e^{\pi \cdot t} \\	% ausgerichtet werden sollen (hier am =-Zeichen...)
\vec{x} &= \begin{pmatrix} 2 \\ 3 \\ -4 \end{pmatrix} \\
A &= \begin{pmatrix} 2 & 3 \\ -4  & 2 \\ 3 & -4 \end{pmatrix}
\end{align*}

Oder mal 'ne Polynomdivision:
\begin{center}
	\polyset{style=C,div=:,vars=x}
	\polylongdiv{x^4-7x^3+3x-2}{x - 2}
\end{center}


\subsubsection{Darstellung von Quellcode}
\begin{lstlisting}
class Fahrzeug{
    String hersteller;
    String farbe;

    //--- Konstruktor, initialisiert die Datenfelder
    Fahrzeug(String derHersteller, String dieFarbe){
        hersteller = derHersteller;
        farbe = dieFarbe;
    }

    //--- Methoden des Fahrzeugs
    beschleunigen(){
    
    }
    
    bremsen(){
    
    }
    
    links(){
    
    }
    
    rechts(){
    
    }
}
\end{lstlisting}


\subsubsection{Darstellung von Zitaten, Quellen, etc.}
Der folgende Zitierstil\cite{referenzname} wird unter anderem in 
der weltbekannten Zeitschrift \emph{nature} verwendet.
Die Quellen werden numerisch automatisch in der erwähnten Reihenfolge 
im Literaturverzeichnis gelistet.

In \LaTeX\, lassen sich über \emph{Syle-Vorgaben} auch andere Zitierstile 
umsetzen, hier gibt es genügend Möglichkeiten für Mediziner, 
Geisteswissenschaftler oder Juristen. Da muß man möglicherweise etwas
recherchieren.

\subsubsection{Die Datei \emph{literatur.bib}}
In der Datei befinden sich letztlich alle Quellen. Dazu werden 
diese einfach nach einem festgelegten Schema festgehalten und 
\LaTeX{} erstellt daraus das Literaturverzeichnis (siehe Ende 
des Dokuments). Der Aufbau ist im Prinzip wie folgt:

Am Besten gleich mal mit dem Literaturverzeichnis vergleichen (in normalem
Editor öffnen)!


\subsubsection{Zitate aus anderen Werken}
Man kann in einer wissenschaftlichen Arbeit durchaus per 
„Copy \& Paste“ einen Text übernehmen. Dieser muß dann allerdings 
mit Angabe der Quelle als Zitat (z.\,B. beidseitig eingerückt) 
dargestellt werden. 

\begin{quote}
	Beispiel für ein Zitat (quote). Es gibt aber auch (quotation) 
	für längere Zitate 	oder sogar (verse) für Verse... Man erkennt 
	hier die beidseitige Einrückung.
\end{quote}

\subsubsection{Gedichte in Versform}
Wie wäre es mit Gedichten?
\begin{verse}
	Ein jeder Stier hat oben vorn\\
	auf jeder Seite je ein Horn;\\
	doch ist es ihm nicht zuzumuten,\\
	auf so ’nem Horn auch noch zu tuten.\\
	Nicht drum, weil er nicht tuten kann,\\
	nein, er kommt mit dem Maul nicht ’ran!.	
\end{verse}

\subsubsection{Farbhervorhebungen}
Texte lassen sich natürlich auch einfärben, das kann 
z.\,B. bei der Beschreibung von Batterieanschlüssen hilfreich sein: 
\textcolor{red}{Plus} ... \textcolor{blue}{Minus}.

\subsubsection{Bilder}
Bilder, Tabellen, etc. werden in der Regel in einer \emph{figure-Umgebung}
gesetzt. Diese ermöglicht einem beispielsweise auch den Verweis (siehe
Abb.: \ref{sportplatz}, siehe auch Quelltext!) auf ein Bild im laufenden 
Text.
\begin{figure}[hbt!]
  \centering
  \includegraphics[width=12cm]{./bilder/wald.jpg}
  \caption{Satellitenaufnahme der Untersuchungsfläche}
  \label{sportplatz}
\end{figure}

\subsubsection{Tabellen}

\begin{table}[hbt!]
  \centering
  \begin{tabular}{ll|c|ll}
    Name  	& Vorname 	& Alter & Geburtstag  	& Geburtsort\\ \hline
    Katrin	& Lollipop  & 16   	& 1.1.1999		& Velbert	\\
    Uschi	& Lomboku	& 17	& 1.1.1998		& Essen		\\
    Pauline	& Fümmli	& 17	& 12.5.1999		& Zürich	\\
  \end{tabular}
\end{table}

\subsubsection{Erweiterungen}
Es gibt für \LaTeX{} wirklich alles Mögliche in Form von Zusatzpaketen, 
die am Anfang des Dokuments mit dem Befehl \textbackslash usepackage\{...\}
importiert werden können. Es gibt z.\,B. Pakete für Kochbücher, Puzzle, 
Kreuzworträtsel, elektronische Schaltsymbole und weiß der Geier was noch.


\section{Vorbereitungen zum Schreiben einer Facharbeit}
Während der Orientierungsphase ist es sehr wichtig, sich Notizen zu machen. Jede 
gesammelte Information sollte in Stichworten festgehalten werden und zwar so, 
daß sie jederzeit wiedergefunden werden kann.
In einer PDF aus dem Internet habe ich beispielsweise Informationen zur 
Typographie entdeckt, die mir ganz interessant erschienen. Also erstelle ich
mir eine entsprechende Notiz mit Informationen, die später evtl. nützlich
sein könnten. Ich weiß ja jetzt noch nicht, ob ich davon etwas verwenden 
möchte und dies am Ende in den Quellen anzugeben gedenke. 

Meine ersten Notizen könnten also etwa so aussehen:
\begin{tcolorbox}[colback=titlepagecolor!5!white,colframe=titlepagecolor!75!black,title=Notizen]
PDF-Datei: Christoph Bier - „typokurz – Einige wichtige typografische Regeln“ - 
Könnte für später ganz interessant sein \\(Datei: /home/Facharbeit/PDFs/
typokurz.pdf); Autor: Christoph Bier, weitere Infos im PDF
\tcblower
Internetseite: https://www.overleaf.com ... Ein Onlineeditor. Vielleicht irgendwo
erwähnen für Leute, die \LaTeX{} nicht installieren können? 
\end{tcolorbox}
Nach und nach taucht man immer tiefer in das Thema ein und nach einiger Zeit 
kann man über viele Aspekte bereits frei reden und hat Zusammenhänge verstanden. 
Dinge, die anfangs wichtig erschienen wurden zwischenzeitlich verworfen, dafür 
kamen neue, andere Aspekte hinzu. 

20\,\%








% ------------------- Erklärung --------------------------------

\newpage

\section*{Erklärung}  % durch das Sternchen fällt die Nummer weg
Hiermit erkläre ich, dass ich die vorliegende Facharbeit selbstständig 
angefertigt, keine anderen als die angegebenen Hilfsmittel benutzt und 
die Stellen der Facharbeit, die im Wortlaut oder im wesentlichen Inhalt 
von anderen Autoren übernommen wurden, mit genauer Quellenangabe kenntlich 
gemacht habe.\\

\vspace{2cm}

\noindent
Velbert, den \today

\hspace*{\fill}
\begin{tabular}{@{}l@{}}\hline
\makebox[6cm]{Vorname Nachname}
\end{tabular}



% ------------------- Literatur / Quellen -----------------------
% Hier werden die Referenzen, welche mit % \cite[S. 101-107]{rferenzname} 
% eingetragen werden geschrieben.

%\newpage
%\bibliography{literatur.bib}


%\addcontentsline{toc}{section}{Literatur}
%\setlength\bibitemsep{6pt}  % Abstand zwischen 2 Einträgen im LitVZ
%\setlength{\bibhang}{2em} % Einrücken 2. Zeile im LitVZ
%


\end{document}